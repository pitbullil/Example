\chapter{Conclusions}\label{chap:Conclusions}

%%%%%%%%%%%%%%%%%%%%%%%%%%%%%%%%%%%%%%%%%%
%%%%%%%  section: Conclusion
%%%%%%%%%%%%%%%%%%%%%%%%%%%%%%%%%%%%%%%%%%
This paper has introduced a novel approach for finding correspondences between shapes in 3D.
This approach is based on a simple observation:
Statistical properties the nearest-neighbor field from the source surface to the target provide robust information about the correspondence.
In particular, we use the diversity of the nearest-neighbor field and the consistency of the internal distances within the surface of corresponding points. 
Two additional ideas of our approach are the use of small sub-surfaces when computing the similarity (rather than using the whole surface) and utilizing a multi-scale approach.

Our approach improves the state-of-the-art results both quantitatively and qualitatively on the challenging benchmarks of SHREC'16.
In particular, these benchmarks contain examples having large deformations, symmetries, partiality of the shapes, and topological noise.
We have demonstrated that our method is robust to the scale of partiality, as well as to its own parameters.
