\chapter{Similarity between sub-surfaces}\label{sec:similarity}

%%%%%%%%%%%%%%%%%%%%%%%%%%%%%%%%%%%%%%%%%%%%%%%%%%%%%%%%%
%%%%%%%%% section: Similarity
%%%%%%%%%%%%%%%%%%%%%%%%%%%%%%%%%%%%%%%%%%%%%%%%%%%%%%%%%

%Problem definition
This section elaborates on Step~3, which is the core of the algorithm.
Given a pair of patches of the same scale, $Q \subset \mathcal{M}$ and $P \subset \mathcal{N}$, this section defines a similarity function between them. 
We require that this function be oblivious to non-rigid deformations, to different resolutions of the meshes, to noise, to topological noise, and to partiality of the data.

%Key idea
We would like to reward a correspondence for which the {\em nearest-neighbor (NN)} field satisfies two properties: it has high diversity of the corresponding points, as well as low inconsistency of the distances.
In what follows we explain these two properties.

\paragraph{Diversity}
When $Q$ and $P$ correspond, each point  on $Q$ should have a unique NN-match on $P$.
Conversely, if $Q$ and $P$ do not correspond, most of the points on $Q$  do not have a good match on $P$.
In the latter case, the nearest neighbors are likely to belong to a small set of points that happen to be somewhat similar to the points of $Q$.
This implies that if the patches correspond, their NN-field is highly \textit{diverse}, i.e., pointing to many different points in $P$.

An intuitive and efficient way to measure diversity is to count the number of unique nearest neighbors between the points of  $Q$ and $P$:
\begin{equation}
	\label{eq:diversity}
	Div(Q,P)=|\{p_i\in P:\exists q_j\in Q,NN(q_j,P)=p_i\}|,
\end{equation}
where  $\{p_i\}_{i=1}^{|P|}$ and $\{q_j\}_{j=1}^{|Q|}$ are the set of points of $Q$ and $P$, respectively and the nearest neighbors is computed between the descriptors (FPFH) of the points.
However, we will see below that the diversity can be calculated implicitly.

\paragraph{Distance inconsistency}
A relatively stable property of  deformed surfaces is their geodesic distances.
Therefore, if two patches correspond, points that belong to a nearest neighbor pair tend to have similar geodesic distances to the centers of the patches they reside on.
Conversely, arbitrary matches typically do not hold this relation.

To realize this observation, we define $DistInconst(Q,P)$, the inconsistency of  $Q$ and $P$, as follows.
Let $p \in P$ and $q \in Q$ be the centers of $P$ and $Q$, respectively;
furthermore, let $p_i= NN(q_j,P)$ be the nearest-neighbor of  $q_j \in Q$.
The deformation implied by the NN-Field for ${p_i,q_j}$ is defined by:
\begin{eqnarray}
	\label{eq:Def}
	&&DistInconst(q_j,p_i,Q,P)= \\
	&&|{GeodDist(q_j,q)-GeodDist(p_i,p)}|/\epsilon, \nonumber
\end{eqnarray}
where $0<\epsilon << Area(\mathcal{M})$ is a used for numerical stability.
We note that diversity was defined between patches, whereas inconsistency was defined between points; this will be clarified later, when we show how to use these ideas within our general similarity function.


\paragraph{Similarity function}
Next, we should integrate the above considerations within a similarity definition, such that similar surfaces will have high diversity and small distance inconsistency.
%
For each $p_i \in P$, we find the minimal inconsistency 
$$r_i^*=min_{q_j \in Q}DistInconst(q_j,p_i,Q,P),$$
such that $p_i$ is the nearest neighbor of $q_j$ in the descriptor (FPFH) space.
Note that some points in $P$ might not be associated with any point in $Q$, since they are not nearest neighbors of any point $q_j\in Q$;
in this case we set $r_i^*=\infty$, in order to make the contribution of $p_i$ be zero.

Finally, we define the similarity between patches $p$ and $Q$ as:
\begin{equation}
	Similarity(P,Q)=\sum_{p_i\in P}\frac{1}{1+r_i^*}.
	\label{eq:DDIS}
\end{equation}
%
It is easy to see that this function rewards low inconsistency.
However, why does it also reward high diversity of the NN-Field?
To understand this, consider the special case where $r_i^*\in\{0,\infty\}$.
When this occurs, the value of Equation~\ref{eq:DDIS} is either $1$ (if $r_i*=0$) or $0$ (if $r_i*=\infty$).  
In the former case, this indicates that a point $p_i$ has a point in $Q$ that considers $p_i$ to be its nearest neighbor.
In this scenario, the similarity function simply counts the number of points in $P$ that are nearest neighbors of some point in $Q$.
But, this is precisely the diversity function we seek-after.

In the general case, the contribution of every point is inversely weighted by its inconsistency $r_i^*$.
This gives preference to NN Fields that preserve pair-wise distances.
