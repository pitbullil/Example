התאמה בין צורות הינה בעיה יסודית בראייה  ממוחשבת ובגרפיקה ממוחשבת, הן בדו מימד, והן בתלת מימד. אפליקציות רבות, כגון הנפשה, שחזור וניתוח צורה דורשות התאמה מדוייקת. עבודה זאת מתמקדת במציאת התאמות בין משטחים בתלת מימד. 

מציאת התאמות בין צורות בתלת מימד הינה משימה מאתגרת, אפילו כאשר העצמים קשיחים ושלמים. בעבודה זו לעומת זאת, אנו עוסקים במציאת התאמות, כאשר מתווספים האתגרים הבאים:
\begin{enumerate}
    \item העצמים עברו התמרות לא קשיחות.
    \item צורה אחת נתונה רק באופן חלקי, ויש להתאימה לגרסה מלאה שלה – בעיה הידועה בשם חלקיות.
    \item 	חלקים לא צמודים של המשטח מצטלבים, דבר הידוע בכינוי רעש טופולוגי.
\end{enumerate}
כל המקרים לעיל מופיעים לעתים תכופות ובצורה תדירה בעולם האמיתי עקב תנאי הרכישה של מידע תלת מימדי.

כבעיה יסודית בתחום הגרפיקה הממוחשבת, קיים גוף נרחב של עבודות המתייחס לפתרונה. שיטות קיימות מתמקדות במזעור של קריטריון עיוות כלשהו, של מתארי צורה מקומית, או של יחסים בין זוגות של נקודות או שילוב בין השניים. למרות שתוצאות מרשימות הושגו על ידי שיטות קיימות, בעיות מסויימות עדיין קיימות. בעיות אילו ניכרות במיוחד בשלושת המקרים הנזכרים לעיל, בייחוד כאשר העיוות קיצוני, חלקיות המודל ניכרת ובמקרים מרובים של רעש טופולוגי.

מספר שיטות שפותחו לאחרונה משתמשות ברשתות נוירונים עמוקות. גישות אלו הראו תוצאות מבטיחות על כמה מדדים של התאמה בין צורות מלאות, ובמקרים מסויימים אף על מדדי התאמה חלקית. אנו נראה ששיטתנו משיגה תוצאות טובות יותר משיטות אלו. בנוסף, כל השיטות המשתמשות בלמידה דורשות קיום של מספר לא מבוטל של מודלים מתוייגים לאימון, דבר שהשגתו הינה אתגר קשה בפני עצמו לעת עתה.

האלגוריתם המוצע בעבודה זו שייך לקטגוריה של העבודות אשר אינן משתמשות בלמידה עמוקה. מוצג בו מדד דמיון חדש, המתבסס על ניתוח שדה השכן הקרוב ביותר בין צורות מקומיות. דהיינו, לכל קודקוד במשטח המקור, אנו מוצאים את השכן הקרוב ביותר במשטח המטרה, במובן של מאפיינים של הצורה מקומית ופונקציות מרחק בין מאפיינים אלו. במקום למזער פונקציה של המרחקים ביניהם, אנו מנתחים את התכונות הסטטיסטיות של השדה הזה. האופי הסטטיסטי של שיטתנו מאפשר לה להתעלם מיוצאי דופן, אשר מהווים מקור לאי אמינות בשיטות אחרות. בפרט, הסטטיסטיקות אותן אנו מציעים נוגעות לשתי תכונות של שדה השכן הקרוב ביותר: (1) גיווניות השדה, כלומר, כמה קודקודים שונים במשטח המטרה מותאמים על ידי השדה, ו-(2) שימור מרחקים של זוגות התאמות משדה השכן הקרוב בין משטח אחד לשני.

בחנו את שיטתנו על שני מסדי נתונים מאתגרים, האחד מכיל צורות חלקיות שעברו התמרה לא קשיחה, והשני מכיל צורות שעברו התמרה לא קשיחה בנוסף לרעש טופולוגי. אנו מדגימים את יתרון שיטתנו הן בצורה איכותנית, והן בצורה כמותית על פני שיטות קיימות. בפרט, שיטתנו משיגה שיפור של עשרה עד עשרים אחוזים בביצועים על פני שיטות מתחרות על מסד הנתונים הראשון (חלקיות) והינה תחרותית עם השיטה הטובה ביותר על מסד הנתונים השני (רעש טופולוגי). בנוסף, אנו מראים בצורה איכותנית את ביצועי שיטתנו על מצבור סריקות אמיתיות של אנשים בתנוחות שונות.

לפיכך תרומתנו בעבודה זו היא כפולה:
\begin{enumerate}
\item	אנו מציגים שיטה חדשה למציאת התאמות בין שני משטחים נתונים, כזו שחסינה להתמרות לא קשיחות שונות, חלקיות צורה ורעש טופולוגי. החדשנות של גישתנו נעוצה בהסתמכותה על תכונותיו הסטטיסטיות של שדה השכן הקרוב ביותר.
\item	אנו מדגימים את יתרונותיה של שיטתנו בצורה נרחבת על מסדי נתונים מאתגרים נפוצים, הן בצורה איכותנית, והן בצורה כמותית.
\end{enumerate}
